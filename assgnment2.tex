\documentclass[journal,12pt,twocolumn]{IEEEtran}
\usepackage{amsthm}
\usepackage{gensymb}
\usepackage{setspace}
\singlespacing
\usepackage[cmex10]{amsmath}
\usepackage{bm}

\usepackage{cases}
\usepackage{mathrsfs}
\usepackage{cite}
\usepackage{stfloats}
\usepackage{mathtools}
\usepackage[breaklinks=true]{hyperref}
\usepackage{graphicx}
\usepackage{subfig}
\usepackage{txfonts}
\usepackage{longtable}
\usepackage{multirow}
\usepackage{tfrupee}
\usepackage{enumitem}
\usepackage{tikz}
\usepackage{steinmetz}
\usepackage{verbatim}
\usepackage{circuitikz}
\usepackage{tkz-euclide}





\usetikzlibrary{calc,math}
\usepackage{listings}
    \usepackage{color}                                            %%
    \usepackage{array}                                            %%
    \usepackage{longtable}                                        %%
    \usepackage{calc}                                             %%
    \usepackage{multirow}                                         %%
    \usepackage{hhline}                                           %%
    \usepackage{ifthen}                                           %%
    \usepackage{lscape}     
\usepackage{multicol}
\usepackage{chngcntr}

\DeclareMathOperator*{\Res}{Res}

\renewcommand\thesection{\arabic{section}}
\renewcommand\thesubsection{\thesection.\arabic{subsection}}
\renewcommand\thesubsubsection{\thesubsection.\arabic{subsubsection}}

\renewcommand \thesectiondis{\arabic{section}}
\renewcommand\thesubsectiondis{\thesectiondis.\arabic{subsection}}
\renewcommand\thesubsubsectiondis{\thesubsectiondis.\arabic{subsubsection}}


\hyphenation{op-tical net-works semi-conduc-tor}
\def\inputGnumericTable{}                                 %%

\lstset{
%language=C,
frame=single, 
breaklines=true,
columns=fullflexible
}
\begin{document}

\newcommand{\BEQA}{\begin{eqnarray}}
\newcommand{\EEQA}{\end{eqnarray}}
\newcommand{\define}{\stackrel{\triangle}{=}}
\bibliographystyle{IEEEtran}
\raggedbottom
\setlength{\parindent}{0pt}
\providecommand{\mbf}{\mathbf}
\providecommand{\pr}[1]{\ensuremath{\Pr\left(#1\right)}}
\providecommand{\qfunc}[1]{\ensuremath{Q\left(#1\right)}}
\providecommand{\sbrak}[1]{\ensuremath{{}\left[#1\right]}}
\providecommand{\lsbrak}[1]{\ensuremath{{}\left[#1\right.}}
\providecommand{\rsbrak}[1]{\ensuremath{{}\left.#1\right]}}
\providecommand{\brak}[1]{\ensuremath{\left(#1\right)}}
\providecommand{\lbrak}[1]{\ensuremath{\left(#1\right.}}
\providecommand{\rbrak}[1]{\ensuremath{\left.#1\right)}}
\providecommand{\cbrak}[1]{\ensuremath{\left\{#1\right\}}}
\providecommand{\lcbrak}[1]{\ensuremath{\left\{#1\right.}}
\providecommand{\rcbrak}[1]{\ensuremath{\left.#1\right\}}}
\theoremstyle{remark}
\newtheorem{rem}{Remark}
\newcommand{\sgn}{\mathop{\mathrm{sgn}}}
\providecommand{\abs}[1]{\vert#1\vert}
\providecommand{\res}[1]{\Res\displaylimits_{#1}} 
\providecommand{\norm}[1]{\lVert#1\rVert}
%\providecommand{\norm}[1]{\lVert#1\rVert}
\providecommand{\mtx}[1]{\mathbf{#1}}
\providecommand{\mean}[1]{E[ #1 ]}
\providecommand{\fourier}{\overset{\mathcal{F}}{ \rightleftharpoons}}
%\providecommand{\hilbert}{\overset{\mathcal{H}}{ \rightleftharpoons}}
\providecommand{\system}{\overset{\mathcal{H}}{ \longleftrightarrow}}
	%\newcommand{\solution}[2]{\textbf{Solution:}{#1}}
\newcommand{\solution}{\noindent \textbf{Solution: }}
\newcommand{\cosec}{\,\text{cosec}\,}
\providecommand{\dec}[2]{\ensuremath{\overset{#1}{\underset{#2}{\gtrless}}}}
\newcommand{\myvec}[1]{\ensuremath{\begin{pmatrix}#1\end{pmatrix}}}
\newcommand{\mydet}[1]{\ensuremath{\begin{vmatrix}#1\end{vmatrix}}}
\numberwithin{equation}{subsection}
\makeatletter
\@addtoreset{figure}{problem}
\makeatother
\let\StandardTheFigure\thefigure
\let\vec\mathbf
\renewcommand{\thefigure}{\theproblem}
\def\putbox#1#2#3{\makebox[0in][l]{\makebox[#1][l]{}\raisebox{\baselineskip}[0in][0in]{\raisebox{#2}[0in][0in]{#3}}}}
     \def\rightbox#1{\makebox[0in][r]{#1}}
     \def\centbox#1{\makebox[0in]{#1}}
     \def\topbox#1{\raisebox{-\baselineskip}[0in][0in]{#1}}
     \def\midbox#1{\raisebox{-0.5\baselineskip}[0in][0in]{#1}}
\vspace{3cm}
\title{Assignment2}%number
\author{CS20Btech11035 -NYALAPOGULA MANASWINI}
\maketitle
\newpage
\bigskip

\renewcommand{\thefigure}{\theenumi}
\renewcommand{\thetable}{\theenumi}
Download python code from 
\begin{lstlisting}
https://github.com/N-Manaswini23/Assignment-2/blob/main/assignment2%20(3).py
\end{lstlisting}
%
Download latex code from
\begin{lstlisting}
https://github.com/N-Manaswini23/Assignment-2/blob/main/assgnment2.tex
\end{lstlisting}
%
\section*{GATE EC QUESTION 63}
Suppose X is a real-valued random variable.Which of the following values CANNOT be attained by$ E[X] $ and $ E[X^2] $ respectively?
\begin{enumerate}
\item[(A)] 0 and 1   
\item[(B)] $\frac{1}{2}$ and $\frac{1}{3}$
\item[(C)] 2 and 3
\item[(D)] 2 and 5
\end{enumerate}

\section*{SOLUTION}
For uniform distribution X,
\begin{align}
E[X^k]&= \int_{-\infty}^{\infty} x^k \mathrm{dF_X(x)} \label{eq:1} \\
var(X)&= \int_{-\infty}^{\infty} (X-E[X])^2\mathrm{dF_X(x)} \\
&=\int_{-\infty}^{\infty}X^2\mathrm{dF_X(x)}  - 2E[X]\int_{-\infty}^{\infty}X \mathrm{dF_X(x)}   \notag \\
& +(E[X])^2\int_{-\infty}^{\infty} \mathrm{dF_X(x)} \\
&=E[X^2]-2(E[X])^2+(E[X])^2 \\
var(X)&=E[X^2]-(E[X])^2 \label{1}
\end{align}
From \eqref{1} , we can conclude that 
\begin{align}
var(X) \geq 0 \label{2}
\end{align}
From \eqref{2}
\begin{align}
E[X^2]-(E[X])^2 & \geq 0
\end{align}
\begin{enumerate}
\item[(A)] $E[X]=0$ and $E[X^2]=1$
\begin{align}
E[X^2]-(E[X])^2 &=1 - 0\\
&=1\\
\therefore E[X^2]-(E[X])^2 &\geq 0
\end{align}
$\therefore$ $E[X]=0$ and $E[X^2]=1$ can be attained \\
\item[(B)] $E[X]=\frac{1}{2}$ and $E[X^2]=\frac{1}{3}$
\begin{align}
E[X^2]-(E[X])^2 &=\frac{1}{3} - \frac{1}{4}\\
&=\frac{1}{12}\\
\therefore E[X^2]-(E[X])^2 &\geq 0
\end{align}
$\therefore$ $E[X]=\frac{1}{2}$ and $E[X^2]=\frac{1}{3}$ can be attained \\
\item[(C)] $E[X]=2$ and $E[X^2]=3$
\begin{align}
E[X^2]-(E[X])^2 &=3 - 4\\
&=-1\\
\therefore E[X^2]-(E[X])^2 &\leq 0
\end{align}
$\therefore$ $E[X]=2$ and $E[X^2]=3$ cannot be attained \\
\item[(D)] $E[X]=2$ and $E[X^2]=5$
\begin{align}
E[X^2]-(E[X])^2 &=5 - 4\\
&=1\\
\therefore E[X^2]-(E[X])^2 &\geq 0
\end{align}
$\therefore$ $E[X]=2$ and $E[X^2]=5$ can be attained\\ \\
$\therefore$ $E[X]=2$ and $E[X^2]=3$ cannot be attained\\
\end{enumerate}
\end{document}
